%%%%%%%%%%%%%%%%%%%%%%%%%%%%%%%%%%%%%%%%%
% Medium Length Professional CV
% LaTeX Template
% Version 3.0 (December 17, 2022)
%
% This template originates from:
% https://www.LaTeXTemplates.com
%
% Author:
% Vel (vel@latextemplates.com)
%
% Original author:
% Trey Hunner (http://www.treyhunner.com/)
%
% License:
% CC BY-NC-SA 4.0 (https://creativecommons.org/licenses/by-nc-sa/4.0/)
%
%%%%%%%%%%%%%%%%%%%%%%%%%%%%%%%%%%%%%%%%%

%----------------------------------------------------------------------------------------
%	PACKAGES AND OTHER DOCUMENT CONFIGURATIONS
%----------------------------------------------------------------------------------------

\documentclass[
	%a4paper, % Uncomment for A4 paper size (default is US letter)
	12pt, % Default font size, can use 10pt, 11pt or 12pt
]{resume} % Use the resume class

% \usepackage{ETbb} % Use the EB Garamond font
\usepackage{ebgaramond}
\usepackage{xcolor}
\usepackage{latexsym}
%------------------------------------------------
\newcommand\CC{$\textsf{CC}_{<:\Box}~$}
\name{Zhaoyi (August) Ge} % Your name to appear at the top

% You can use the \address command up to 3 times for 3 different addresses or pieces of contact information
% Any new lines (\\) you use in the \address commands will be converted to symbols, so each address will appear as a single line.

\address{Waterloo, ON} % Main address

\address{z33ge@uwaterloo.ca} % Contact information

%----------------------------------------------------------------------------------------

\begin{document}

%----------------------------------------------------------------------------------------
%	EDUCATION SECTION
%----------------------------------------------------------------------------------------

\begin{rSection}{Education}
	\textbf{University of Waterloo} \hfill {September 2020 - May 2025} \\ 
	Bachelor of Computer Science (GPA: 91\%)
\end{rSection}

\begin{rSection}{Research Interests}
	Type Systems / Functional Programming / Logic in Computer Science.
\end{rSection}

\begin{rSection}{Publications}
	{Cong Ma, \textbf{Zhaoyi Ge}, Edward Lee, Yizhou Zhang}. \textcolor{purple}{\textbf{Lexical Effect Handlers, Directly}}.
	\textit{39th ACM International Conference on Object-Oriented Programming, Systems, Languages, and Applications}.
	\\
	\textbf{Conditionally Accepted, OOPSLA 2024}.
\end{rSection}

\begin{rSection}{Research Experience}
	\begin{rSubsection}{Univeristy of Waterloo}{September 2023 - December 2023, May 2024 - Present}{Undergraduate Research Fellow}{}
		\item Supervised by: Yizhou Zhang.
		\item Constructed a compiler for Lexi, a language with a compilation strategy that supports
		efficient lexical effect handlers. The compiler is written in OCaml and compiles Lexi into C.
		The compiler features closure conversion and tail call optimization. 
		Rewrote suite of effect handlers benchmark for using Lexi.
		\item Constructed a compiler for STAL (Stack-based Typed Assembly Language) using OCaml and x86 Assembly,
		implemented a type checker for assembly instructions and stack/heap operations. 
	\end{rSubsection}
	\begin{rSubsection}{Univeristy of Waterloo}{May 2024 - Present}{Undergraduate Research Assistant}{}
		\item Supervised by: Ondřej Lhoták.
		\item Refactored the Coq soundness proof for System \CC by using inductively defined capture sets 
		instead of a set-based definition, which simplifies the proof 
		and provides the potential for extensions to capture sets.
	\end{rSubsection}
\end{rSection}

%----------------------------------------------------------------------------------------
%	WORK EXPERIENCE SECTION
%----------------------------------------------------------------------------------------

\begin{rSection}{Work Experience}

	\begin{rSubsection}{Genesys}{May 2023 - August 2023}{Software Developer Intern}{Markham, Ontario}
		\item Led the development of a security automation service in Python for cloud native applications.
		\item Designed DynamoDB schemas for efficient data lookup.
		% \item Designed DynamoDB table to enable fast data retrieval with indexing.
	\end{rSubsection}

%------------------------------------------------

	\begin{rSubsection}{MeshAI}{May 2022 - August 2022}{Software Developer Intern}{Remote}
		\item Built healthcare services using Java, GraphQL and Vue.js. Improved response time by caching using Redis.
		% \item Developed features for a serverless web application using GraphQL and AWS Amplify.
	\end{rSubsection}

%------------------------------------------------

	% \begin{rSubsection}{Xinfangde Technology Group Co., Ltd.}{September 2021 - December 2021}{Full-stack Developer Intern}{China}
	% 	\item Implemented algorithms and APIs for quantitative analysis of mutual funds using Java.
	% \end{rSubsection}

\end{rSection}

\begin{rSection}{Projects}
     \begin{rSubsection}{Proust}{CS 245E Course Project}{}{}
        \item Developed a simple interactive \textbf{proof assistant} for propositional and predicate logic using Racket.
        \item Proved theorems about natural numbers and Boolean algebra using the proof assistant.
     \end{rSubsection}
     \begin{rSubsection}{Lacs}{CS 241E Course Project}{}{}
		\item Implemented a \textbf{compiler} for a minimal Scala-like language using Scala.
        \item Developed features such as garbage collection, higher-order functions and closures.
	\end{rSubsection}
 % \begin{rSubsection}{DAVIS Video Segmentation Challenge}{CS 484 Course Project}{}{}
 %    \item Implemented a convolutional neural networks video segmentation model in PyTorch, achieved a score of 60\% on the evaluation dataset.
 % \end{rSubsection}
\end{rSection}


%----------------------------------------------------------------------------------------
%	TECHNICAL STRENGTHS SECTION
%----------------------------------------------------------------------------------------

\begin{rSection}{Technical Skills}

	\begin{tabular}{@{} >{\bfseries}l @{\hspace{6ex}} l @{}}
		% Computer Languages & Prolog, Haskell, AWK, Erlang, Scheme, ML \\
  	Languages & OCaml, Coq, Scala, Agda, Racket, C++.\\
	Tools and Technologies & Git, Docker, x86 Assembly.
	\end{tabular}

\end{rSection}

%----------------------------------------------------------------------------------------
%	EXAMPLE SECTION
%----------------------------------------------------------------------------------------

%\begin{rSection}{Section Name}

	%Section content\ldots

%\end{rSection}

%----------------------------------------------------------------------------------------

\end{document}
