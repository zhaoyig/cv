%%%%%%%%%%%%%%%%%%%%%%%%%%%%%%%%%%%%%%%%%
% Medium Length Professional CV
% LaTeX Template
% Version 3.0 (December 17, 2022)
%
% This template originates from:
% https://www.LaTeXTemplates.com
%
% Author:
% Vel (vel@latextemplates.com)
%
% Original author:
% Trey Hunner (http://www.treyhunner.com/)
%
% License:
% CC BY-NC-SA 4.0 (https://creativecommons.org/licenses/by-nc-sa/4.0/)
%
%%%%%%%%%%%%%%%%%%%%%%%%%%%%%%%%%%%%%%%%%

%----------------------------------------------------------------------------------------
%	PACKAGES AND OTHER DOCUMENT CONFIGURATIONS
%----------------------------------------------------------------------------------------

\documentclass[
	%a4paper, % Uncomment for A4 paper size (default is US letter)
	12pt, % Default font size, can use 10pt, 11pt or 12pt
]{resume} % Use the resume class

% \usepackage{ETbb} % Use the EB Garamond font
\usepackage{ebgaramond}

%------------------------------------------------

\name{Zhaoyi (August) Ge} % Your name to appear at the top

% You can use the \address command up to 3 times for 3 different addresses or pieces of contact information
% Any new lines (\\) you use in the \address commands will be converted to symbols, so each address will appear as a single line.

\address{Waterloo, ON} % Main address

\address{z33ge@uwaterloo.ca} % Contact information

%----------------------------------------------------------------------------------------

\begin{document}

%----------------------------------------------------------------------------------------
%	EDUCATION SECTION
%----------------------------------------------------------------------------------------

\begin{rSection}{Education}
	\textbf{University of Waterloo} \hfill {September 2020 - May 2025} \\ 
	Bachelor of Computer Science (GPA: 91\%) \\
    Relevant Courses: \\
    \hspace*{5mm} CS 245E - Logic and Computation (Enriched) - 100\% \\
    \hspace*{5mm} CS 241E - Foundations of Sequential Programs (Enriched) - 91\% \\
    \hspace*{5mm} PMATH 347 - Groups and Rings - 93\% 
\end{rSection}

\begin{rSection}{Research Interests}
	Programming language design/ Type system/ Semantics/ Logic in computer science.
\end{rSection}

\begin{rSection}{Research Experience}
	\begin{rSubsection}{Univeristy of Waterloo}{September 2023 - December 2024}{Undergraduate Research Fellow}{}
		\item \textbf{Supervisor}: Yizhou Zhang
		\item \textbf{Project Title}: SSTAL: Stack-based Typed Assembly Language with Multi-stack Semantics
		\item The aim of this project was to develop an efficient and type safe target language for high-level languages with lexical effect handlers.
		The target language allows fast effect handling by having a type-safe multi-stack hierarchy.
		I implemented a prototype compiler of SSTAL in OCaml, which invovles designed and implemented type-checking algorithms for stack types and capabilities.
		This work is to be submitted to ICFP 2024.
		% stack type checking algorithm, determinstic
		% 
	\end{rSubsection}
\end{rSection}

%----------------------------------------------------------------------------------------
%	WORK EXPERIENCE SECTION
%----------------------------------------------------------------------------------------

\begin{rSection}{Work Experience}

	\begin{rSubsection}{Genesys}{May 2023 - August 2023}{Software Developer Intern}{Markham, Ontario}
		\item Led the development of a security automation service in Python for cloud native applications.
		\item Designed DynamoDB table to enable fast data retrieval with indexing.
	\end{rSubsection}

%------------------------------------------------

	\begin{rSubsection}{MeshAI}{May 2022 - August 2022}{Software Developer Intern}{Remote}
		\item Built healthcare services using Java and GraphQL. Improved response time by caching using Redis.
		\item Developed features for a serverless web application using GraphQL and AWS Amplify.
	\end{rSubsection}

%------------------------------------------------

	% \begin{rSubsection}{Xinfangde Technology Group Co., Ltd.}{September 2021 - December 2021}{Full-stack Developer Intern}{China}
	% 	\item Implemented algorithms and APIs for quantitative analysis of mutual funds using Java.
	% \end{rSubsection}

\end{rSection}

\begin{rSection}{Projects}
     \begin{rSubsection}{Proust}{CS 245E Course Project}{}{}
        \item Developed a simple interactive proof assistant for propositional and predicate logic using Racket.
        \item Proved theorems about natural numbers and Boolean algebra using the proof assistant.
     \end{rSubsection}
     \begin{rSubsection}{Lacs}{CS 241E Course Project}{}{}
		\item Implemented a compiler for a minimal Scala-like language using Scala.
        \item Developed features such as garbage collection, higher-order functions and closures.
	\end{rSubsection}
 % \begin{rSubsection}{DAVIS Video Segmentation Challenge}{CS 484 Course Project}{}{}
 %    \item Implemented a convolutional neural networks video segmentation model in PyTorch, achieved a score of 60\% on the evaluation dataset.
 % \end{rSubsection}
\end{rSection}


%----------------------------------------------------------------------------------------
%	TECHNICAL STRENGTHS SECTION
%----------------------------------------------------------------------------------------

\begin{rSection}{Technical Skills}

	\begin{tabular}{@{} >{\bfseries}l @{\hspace{6ex}} l @{}}
		% Computer Languages & Prolog, Haskell, AWK, Erlang, Scheme, ML \\
  	Languages & OCaml, Coq, Scala, Agda, Racket, C++.\\
	Tools and Technologies & Git, Docker, x86 Assembly.
	\end{tabular}

\end{rSection}

%----------------------------------------------------------------------------------------
%	EXAMPLE SECTION
%----------------------------------------------------------------------------------------

%\begin{rSection}{Section Name}

	%Section content\ldots

%\end{rSection}

%----------------------------------------------------------------------------------------

\end{document}
